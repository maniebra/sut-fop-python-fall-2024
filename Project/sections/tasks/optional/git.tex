برای این بخش، شما از ابتدا باید برای پروژه‌تان یک مخزن\footnote{\lr{Repository}} \lr{Git} ایجاد نمایید. برای این کار ابتدا وارد پوشه‌ای می‌شوید که در آن بناست پروژه را ذخیره کنید و سپس دستور زیر را در ترمینال یا \lr{CMD} یا \lr{PowerShell} اجرا کنید:

\begin{terminal}
    \begin{lstlisting}[language=bash]
C:/Users/username/Desktop/project> git init
    \end{lstlisting}
\end{terminal}

سپس هربار تغییری در کد می‌دهید می‌توانید به صورت زیر از تغییراتتان ذخیره کنید:

\begin{terminal}
    \begin{lstlisting}[language=bash]
C:/Users/username/Desktop/project> git add .
C:/Users/username/Desktop/project> git commit -m "commit message"
    \end{lstlisting}
\end{terminal}

که در آن \code{commit message} یک متن است که شما می‌توانید برای توضیح تغییراتتان استفاده کنید.

برای مطالعه‌ی بیشتر در مورد \lr{Git} به ضمیمه ۲ مراجعه کنید.

همچنین برای کسب نمره کامل از این بخش، مخزن \lr{Git} شما باید دارای حداقل ۵ تغییر\footnote{\lr{Commit}} باشد.