
در این بخش، شما وظیفه دارید که پروژه خود را بر روی اکانت گیتهاب خود منتشر کنید. برای اینکار،ابتدا نیاز است به وبسایت \href{https://www.github.com}{\lr{GitHub}} بروید و در آنجا اکانت خود را بسازید. پس از ساختن اکانت،وارد آن شوید و مراحل زیر را انجام دهید:

\subsection{روش اول: استفاده از محیط ترمینال}

اگر بخش \lr{Git} را انـجام می‌دهید، توصیه می‌کنیم از دستورات گفته شده در این روش استفاده نمایید.

\begin{enumerate}
    \item \textbf{ساختن مخزن}
    \begin{enumerate}
        \item {روی دکمه \lr{New Repository} کلیک کنید.}
        \item {نامی برای مخزن انتخاب کنید.}
        \item {مخزن را عمومی یا خصوصی تنظیم کنید.}
        \item {روی \lr{Create Repository}  کلیک کنید.}
    \end{enumerate}
    \item \textbf{اتصال مخزن محلی به \lr{GitHub}}:\\\\
    برای اتصال مخزن \lr{Git} محلی به مخزن ایجادشده در \lr{GitHub}، دستور زیر را اجرا کنید:
    
    \begin{terminal}
    \begin{lstlisting}[language=bash]
    git remote add origin https://github.com/username/repository.git
    \end{lstlisting}
    \end{terminal}

    که \code{repository} نام مخزن گیتهاب شما و \code{username} نام کاربری شما در گیتهاب است.

    \item \textbf{ارسال تغییرات به \lr{GitHub}}:\\\\
    برای ارسال تغییرات ثبت‌شده به \lr{GitHub}، دستورات زیر را اجرا کنید
    \begin{terminal}
    \begin{lstlisting}[language=bash]
    git push -u origin main
    \end{lstlisting}
    \end{terminal}  
    
    \item \textbf{کلون کردن مخزن از \lr{GitHub}}:\\\\
    برای دریافت یک مخزن موجود از \lr{GitHub} روی سیستم خود، دستور زیر را اجرا کنید:
    \begin{terminal}
    \begin{lstlisting}[language=bash]
    git clone https://github.com/username/repository.git
    \end{lstlisting}
    \end{terminal} 

     \item \textbf{مدیریت تغییرات در \lr{GitHub}}:\\\\
     برای مشاهده سوابق تغییرات، دستور زیر را اجرا کنید:
     \begin{terminal}
    \begin{lstlisting}[language=bash]
    git log
    \end{lstlisting}
    \end{terminal} 

    \item \textbf{برای همگام‌سازی تغییرات جدید از مخزن \lr{GitHub}، از دستور زیر استفاده کنید:}:\\\\
    \begin{terminal}
    \begin{lstlisting}[language=bash]
    git pull origin main
    \end{lstlisting}
    \end{terminal} 

    \item \textbf{ایجاد شاخه جدید برای ویژگی‌ها}:\\\\
    برای توسعه ویژگی‌های جدید بدون تأثیر بر شاخه اصلی:
    \begin{terminal}
    \begin{lstlisting}[language=bash]
    git branch feature-branch
    git checkout feature-branch
    \end{lstlisting}
    \end{terminal} 
    
    
\end{enumerate}


\subsection{روش دوم: آپلود دستی فایل‌ها}

پس از ایجاد مخزن کد، فایل‌های پروژه‌تان را به صورت دستی (مثلاً با \lr{drag and drop}) به مخزن اضافه کنید.