در این بخش شما موظفید تا ساختار‌های داده‌ای مورد نیاز برای مدل‌ها را با استفاده از لغت‌نامه\footnote{\lr{dictionary}} پیاده‌سازی کنید. این بخش را دقیقا می‌توانید مانند سوال چهارم تمرین ۳ یا سوال آخر تمرین ۴ پیاده کنید؛ به این صورت که مشخصات هر مدل را به صورت یک کلید\footnote{\lr{key}} قرار داده و مقداردهی آن را از طریق مقدار\footnote{\lr{value}} انـجام دهید. همچنین می‌توانید به منظور ذخیره‌سازی تمام این لغت‌نامه‌ها از یک لیست\footnote{\lr{list}} استفاده کنید.

\subsection{مدل دانشجو \lr{(Student)}}

مشخصات یک مدل دانشجو به شکل زیر است:

\begin{table}[H]
    \centering
    \begin{tabular}{|c|c|}
        \hline
        کلید & نوع داده \\
        \hline
        \code{id} & \code{int} \\
        \code{name} & \code{str} \\
        \code{email} & \code{str} \\
        \code{password} & \code{str} \\
        \code{phone} & \code{str} \\
        \hline
    \end{tabular}
\end{table}

\subsection{مدل استاد \lr{(Professor)}}

مشخصات یک مدل استاد به شکل زیر است:

\begin{table}[H]
    \centering
    \begin{tabular}{|c|c|}
        \hline
        کلید & نوع داده \\
        \hline
        \code{id} & \code{int} \\
        \code{name} & \code{str} \\
        \code{email} & \code{str} \\
        \code{password} & \code{str} \\
        \code{phone} & \code{str} \\
        \hline
    \end{tabular}
\end{table}

\subsection{مدل درس \lr{(Course)}}

مدل هر درس به صورت زیر خواهد بود:

\begin{table}[H]
    \centering
    \begin{tabular}{|c|c|}
        \hline
        کلید & نوع داده \\
        \hline
        \code{id} & \code{int} \\
        \code{name} & \code{str} \\
        \code{description} & \code{str} \\
        \code{professor} & \code{int} \\
        \code{students} & \code{list[int]} \\
        \hline
    \end{tabular}
\end{table}

\subsection{شکل نهایی کد شما}

بخشی از کدتان که در آن داده‌ساختار‌ها را ذخیره کرده‌اید، به شکل زیر خواهد بود:

\sourcecode{codes/2.1.py}

توجه کنید که ممکن است به ساختار‌های بیشتری نیاز پیدا کنید که نیازسنجی و پیاده‌سازی آن‌ها بر عهده‌‌ی شماست.