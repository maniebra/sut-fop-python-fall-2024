\documentclass[a4paper,12pt]{article}

\usepackage{fontspec} % Required for XeLaTeX
\usepackage{xepersian} % Persian support for XeLaTeX

\settextfont{Times New Roman} % Default font for text
\setdigitfont{Times New Roman} % Default font for numbers

\setlength{\parindent}{0pt} % Remove paragraph indentation
\setlength{\parskip}{1em}   % Add space between paragraphs
\begin{document}
\maketitle

\section*{پیاده سازی فرآیند ایجاد کلاس و ثبت نام دانشجو}
در این بخش از سیستم، شما وظیقه دارید که کلاس ها را ایجاد و مدیریت کنید. هر کلاس روم اطلاعاتی مانند نام کلاس، مدرس، ظرفیت، و زمان شروع را ذخیره می‌کند و به سایر ماژول‌های سیستم مانند ماژول ثبت‌نام دانشجویان و مدیریت محتوا متصل می‌شود.همچنین هر دانشجو می‌تواند با اضافه کردن آیدی کلاس،به آن اضافه شود.

\section*{ویژگی‌ها}
\begin{enumerate}
    \item \textbf{ایجاد کلاس روم جدید}:
    \begin{itemize}
        \item ایجاد آیدی منحصر به فرد برای کلاس
        \item تعریف نام کلاس.
        \item اختصاص مدرس.
        \item تعیین ظرفیت کلاس.
        \item تنظیم زمان‌های کلاس.
    \end{itemize}
    \item \textbf{ویرایش اطلاعات کلاس روم}:
    \begin{itemize}
        \item امکان تغییر نام، مدرس، ظرفیت، یا زمان‌های کلاس.
    \end{itemize}
    \item \textbf{حذف کلاس روم}:
    \begin{itemize}
        \item حذف کلاس روم از سیستم، با بررسی وابستگی‌ها (مانند دانشجویان ثبت‌نام‌شده).
    \end{itemize}
    \item \textbf{نمایش لیست کلاس روم‌ها}:
    \begin{itemize}
        \item بازیابی و نمایش همه کلاس روم‌ها برای مدیریت بهتر.
    \end{itemize}
\end{enumerate}


\end{document}
