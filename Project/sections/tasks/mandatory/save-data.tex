استاد باید بتواند به طور مشخص اطلاعات مربوط به دانشجویان را ذخیره کند.  
در این بخش این قابلیت را دانشجویان نخواهند داشت (دانشجویان می‌توانند اطلاعات مربوط به درس مشخص و استاد همان درس را ذخیره کنند، اما دسترسی به لیست دانشجویان دیگر نخواهند داشت).  
همچنین استاد توانایی ذخیره کردن تمامی اطلاعات کلاس‌ها را نیز باید داشته باشد و آن‌ها را بر روی یک فایل ذخیره کند. ذخیره آن روی فایل به این معناست که پس از خروج از برنامه، اطلاعات نباید پاک شوند و در ابتدا از روی فایل خوانده شوند. توجه شود که نمرات دانشجویان هر درس نیز باید روی فایل ذخیره شود تا مراحل بعدی ممکن شوند.

راهنمایی:‌ به جای پیاده‌سازی یک سیستم به صورت دستی برای ذخیره داده‌ها می‌توانید از تابع \code{dump}‌ در کتابخانه \code{json} استفاده نمایید. (هدف این بخش نیز یادگیری استفاده از ماژول‌هاست)

تا اینجای کار بخش توابع برنامه شما می‌تواند به صورت زیر باشد:

\sourcecode{codes/2.6.py}