استاد باید بتواند به طور مشخص اطلاعات مربوط به دانشجویان را ذخیره کند.  
در این بخش این قابلیت را دانشجویان نخواهند داشت (دانشجویان می‌توانند اطلاعات مربوط به درس مشخص و استاد همان درس را ذخیره کنند، اما دسترسی به لیست دانشجویان دیگر نخواهند داشت).  
همچنین استاد توانایی ذخیره کردن تمامی اطلاعات کلاس‌ها را نیز باید داشته باشد و آن‌ها را بر روی یک فایل ذخیره کند. ذخیره آن روی فایل به این معناست که پس از خروج از برنامه، اطلاعات نباید پاک شوند و در ابتدا از روی فایل خوانده شوند. توجه شود که نمرات دانشجویان هر درس نیز باید روی فایل ذخیره شود تا مراحل بعدی ممکن شوند.

توجه کنید که صرفاً ذخیره اطلاعات برای اجراهای بعدی برنامه در این بخش معیار نمره گرفتن است و اگر به جای فایل اطلاعات را حتی داخل پایگاه‌های داده مانند \lr{MySQL} یا \lr{MongoDB} یا حتی \lr{SQLite} ذخیره کردید نه تنها موجب کسر نمره نخواهد شد بلکه این امر تشویق نیز می‌شود.