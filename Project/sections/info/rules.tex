\chapter{معرفی پروژه}
\section{شرح پروژه}

در این پروژه قرار است شما یک \lr{LMS} یا سامانه مدیریت یادگیری بسازید، یعنی سامانه‌ای مشابه با \lr{CW} شریف یا \lr{Courses} امیرکبیر که خوب است بدانید برپایه‌ی \lr{Moodle}\footnote{\href{https://moodle.org}{\url{https://moodle.org}}} ساخته شده‌اند. 
در این راستا، شما با رویکرد شی‌گرا مدل‌های سامانه را تعریف خواهید کرد و با ایجاد ارتباط منطقی میان آن‌ها یک سامانه‌ی قابل اجرا خواهید داشت. همچنین در بخش‌هایی از این پروژه از شما خواسته خواهد شد تا با استفاده از \lr{Pandas} و \lr{Numpy} داده‌های کلاسی را پردازش کرده و یک خروجی خوانا برای کاربران سامانه ایجاد کنید.

\section{انتظارات و اهداف}

انتظار می‌رود در انتهای این پروژه:

\begin{itemize}
    \item {
        رویکرد شی‌گرا را به خوبی آموخته و بتوانید با آن یک پروژه‌ی قابل اجرا بسازید.
    }
    \item {
        بتوانید با استفاده از \lr{Pandas} و \lr{Numpy} داده‌ها را پردازش کرده و یک خروجی خوانا برای کاربران سامانه ایجاد کنید.
    }
    \item {
        تا حد بسیار مبتدی با مفاهیم ذخیره‌سازی داده‌ها و پردازش آن‌ها آشنا شوید.
    }
\end{itemize}
\noindent
و همچنین در صورت انجام بخش اختیاری هم از شما انتظار می‌رود:

\begin{itemize}
    \item {
        بتوانید با \lr{Git} تغییرات یک پروژه را مدیریت کنید.
    }
    \item {
        با گیت‌هاب آشنا شوید و بتوانید با آن یک پروژه را مدیریت کنید.
    }
    \item {
        با مفهوم \lr{GUI} آشنا شوید و بتوانید با استفاده از \lr{Tkinter} یا \lr{PyQt} یک پنجره‌ی \lr{GUI} بسازید.
    }
    \item {
        با اهمیت آزمایش آشنا شده و بتوانید برای یک پروژه \lr{Unit Test} بنویسید.
    }
\end{itemize}

\section{قوانین مربوط به پروژه}

\begin{itemize}
    \item {
        مستند پروژه را یک بار تا انتها بخوانید، تعدادی از موارد امتیازی را (در صورتی که تمایل به انـجام آن‌ها دارید) می‌بایست از همان ابتدای شروع پروژه انـجام دهید و در نظر داشته باشید.
    }
    \item {
        تمام فایل‌های پروژه‌تان را در یک فایل فشرده با \underline{فرمت \,\texttt{.zip}}  روی کوئرا بارگذاری کنید.
    }
    \item {
        نام فایلی که بارگذاری می‌شود باید به فرمت \texttt{FOP\_PROJ\_\{STDID1\}\_\{STDID2\}} باشد که \texttt{STDID1} شماره دانشجویی عضو اول گروه و \texttt{STDID2} شماره دانشجویی عضو دوم گروه است.
        برای مثال اگر یک گروه داشته باشیم که شماره دانشجویی اعضایش \texttt{403108123} و \texttt{403108987} باشد،
        باید فایلی به نام \texttt{FOP\_PROJ\_403108123\_403108987.zip} آپلود کنند.\\
        {\large \underline{آپلود هرگونه فایل با نامگذاری خارج از این چارچوب موجب کسر نمره خواهد شد}.}
    }
    \item {
        هر دو عضو گروه می‌بایست فایل مربوطه را بارگذاری نمایند. مسئولیت عدم بارگذاری یا بارگذاری فایل‌های متفاوت توسط اعضای یک گروه متوجه خود ایشان است.
    }
    \item {
        از آنجا که تحویل پروژه اجباری است، مطمئن شوید هر دو عضو گروه کاملاً بر پروژه مسلط باشند. همچنین هر دو عضو موظف به فعالیت هستند و در صورتی که یکی از اعضا فعالیت نکند، تمامی اعضا به عنوان یک گروه دچار کسر نمره خواهد شد.
    }
    \item {
        پروژه تاخیر نخواهد داشت و زمان پایان اعلام شده، نهایی (هارد ددلاین) خواهد بود.
    }
    \item {
        اطمینان حاصل شده است که پروژه‌ی شما با تمامی مطالبی که در کلاس آموخته‌اید قابل انجام باشد. پس پیش از انجام پروژه، همه‌ی مطالبی که در کلاس آموخته‌اید را به خوبی مرور بفرمایید.
    }
\end{itemize}