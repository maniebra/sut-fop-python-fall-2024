\chapter{معرفی پروژه}
\section{شرح پروژه}

\section{انتظارات و اهداف}

\section{قوانین مربوط به پروژه}

\begin{itemize}
    \item {
        مستند پروژه را یک بار تا انتها بخوانید، تعدادی از موارد امتیازی را (در صورتی که تمایل به انـجام آن‌ها دارید) می‌بایست از همان ابتدای شروع پروژه انـجام دهید و در نظر داشته باشید.
    }
    \item {
        تمام فایل‌های پروژه‌تان را در یک فایل فشرده با \underline{فرمت \,\texttt{.zip}}  روی کوئرا بارگذاری کنید.
    }
    \item {
        نام فایلی که بارگذاری می‌شود باید به فرمت \texttt{FOP\_PROJ\_\{STDID1\}\_\{STDID2\}} باشد که \texttt{STDID1} شماره دانشجویی عضو اول گروه و \texttt{STDID2} شماره دانشجویی عضو دوم گروه است.
        برای مثال اگر یک گروه داشته باشیم که شماره دانشجویی اعضایش \texttt{403108123} و \texttt{403108987} باشد،
        باید فایلی به نام \texttt{FOP\_PROJ\_403108123\_403108987.zip} آپلود کنند.\\
        {\large \underline{آپلود هرگونه فایل با نامگذاری خارج از این چارچوب موجب کسر نمره خواهد شد}.}
    }
    \item {
        هر دو عضو گروه می‌بایست فایل مربوطه را بارگذاری نمایند. مسئولیت عدم بارگذاری یا بارگذاری فایل‌های متفاوت توسط اعضای یک گروه متوجه خود ایشان است.
    }
\end{itemize}