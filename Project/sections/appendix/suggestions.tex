\section{گیت \lr{(Git)}}

همانطور که از تمرین صفر به یاد دارید، گیت‌هاب محلی بود که در آن مخازن کد نسخه‌گذاری‌ شده با سیستم کنترل نسخه\footnote{\lr{Version Control System (VCS)}} گیت قرار می‌گرفتند.
برای یادگیری کار با \lr{Git} و \lr{GitHub} توصیه می‌کنیم \href{https://www.youtube.com/watch?v=tRZGeaHPoaw}{\underline{این ویدیو}} و \href{https://www.youtube.com/watch?v=HkdAHXoRtos}{\underline{این ویدیو}} را مشاهده بفرمایید.

\section{شی‌گرایی}

مفهوم شی‌گرایی در کلاس درس برایتان مورد بررسی قرار گرفت. اما برای مرور چند نکته برایتان در ادامه خواهیم آورد.

\vspace{3mm}
به طور کلی، از رویکرد شی‌گرا و برنامه‌نویسی شی‌گرا\footnote{\lr{Object-oriented programming (OOP)}} زمانی استفاده می‌کنیم که می‌خواهیم یک شی از دنیای واقعی را درون یک سیستم یا برنامه مدل کنیم. برای مثال با هم یک کتاب را مدل می‌کنیم.

فرض کنید مشخصه‌های یک کتاب به صورت زیر باشد:

\begin{itemize}
    \item {تعداد صفحات}
    \item {نام مولف}
    \item {باز یا بسته بودن (فرض کنید در ابتدا کتاب بسته است.)}
    \item {صفحه‌ی فعلی (فرض کنید در ابتدای کار کتاب در صفحه‌ی اول خود باشد.)}
\end{itemize}

و همچنین عملیات زیر می‌تواند روی یک کتاب انـجام شود:


\begin{itemize}
    \item {باز کردن کتاب}
    \item {بستن کتاب}
    \item {رفتن به صفحه‌ی \code{p} در کتاب}
\end{itemize}

در این صورت کد زیر را خواهیم داشت:

\sourcecode{codes/4.2.py}

\section{تست واحد (\lr{Unit test})}

به طور کلی تست واحد را برای آزمون عملکرد صحیح و منطقی بخش‌هایی از برنامه‌هایمان می‌نویسیم. فرض کنید یک فایل به نام \code{my\_module.py} نوشته‌ایم که محتوای آن تابع زیر است:

\sourcecode{codes/4.3.1.py}


برای تست آن، از تست واحد (درون فایل \code{test\_my\_module.py}) زیر استفاده می‌کنیم.

\sourcecode{codes/4.3.2.py}

و برای تست، فایل تستمان را اجرا می‌کنیم
که قبول\footnote{\lr{pass}} شدن آن به معنای عملکرد صحیح و به مشکل خوردن آن\footnote{\lr{fail}} به معنای عملکرد نادرست است.


\section{رسم نمودار}

برای یادگیری ماژول‌های \code{matplotlib} و \lr{seaborn} استفاده از دوره‌های آموزشی \href{https://www.kaggle.com}{\underline{\lr{Kaggle}}} یا مشاهده‌‌ی \href{https://www.youtube.com/watch?v=LnGz20B3nTU}{\underline{این ویدیو}} پیشنهاد می‌شود.

\section{رابط کاربری گرافیکی}

برای انجام بخش‌های مربوط به رابط کاربری گرافیکی\footnote{\lr{GUI}} توصیه می‌شود از ماژول \lr{Tkinter} و یا \lr{PyQT} استفاده کنید.

در صورتی که از \lr{Tkinter} استفاده می‌کنید \href{https://www.youtube.com/watch?v=ibf5cx221hk}{\underline{این ویدیو}} و در صورتی که از \lr{PyQT} استفاده می‌کنید \href{https://www.youtube.com/watch?v=MOItX2aKTGc}{\underline{این ویدیو}} و یا \href{https://www.youtube.com/watch?v=hX8fj8SGZJs}{\underline{این ویدیو}} را مشاهده بفرمایید.