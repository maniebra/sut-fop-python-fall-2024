\documentclass{report}[a4paper]

% THIRD PARTY PACKAGES
\usepackage[margin=2cm]{geometry}
\usepackage{fontspec}
\usepackage[many]{tcolorbox}
\usepackage{graphicx}
\usepackage[hidelinks]{hyperref}
\usepackage{varwidth}
\usepackage{xepersian}

% CUSTOM STYLES
% Define colors
\definecolor{warning-bg}{RGB}{255, 253, 205}
\definecolor{warning-border}{RGB}{30, 255, 255}
\definecolor{warning-title}{RGB}{51, 51, 51}

\newtcolorbox{warningbox}[1][]{%
    enhanced,
    colback=warning-bg,
    colframe=warning-border,
    boxrule=0pt,
    borderline west={3pt}{0pt}{warning-border},
    sharp corners,
    title={\large\bfseries\textcolor{black}{⚠\hspace{2mm}توجه!}},
    attach boxed title to top left={
        xshift=0mm,
        yshift*=-\tcboxedtitleheight/2
    },
    boxed title style={
        sharp corners,
        size=small,
        colback=warning-border,
        colframe=warning-border,
    },
    fonttitle=\bfseries,
    coltitle=black,
    before skip=2mm,
    after skip=2mm,
    breakable,
    #1
}


\settextfont{X Yekan}


\begin{document}
\thispagestyle{empty}
\begin{center}
    {دانشگاه صنعتی شریف}\\
    \vspace{8mm}
    \includegraphics[width=5cm]{images/sharif.png}\\
    \vspace{12mm}
    {\large دانشکده مهندسی کامپیوتر}\\
    \vspace{16mm}

    {\huge پروژه درس مبانی برنامه‌سازی پایتون}\\
    \vspace{5mm}
    {\Large گروه ۷ - پاییز ۱۴۰۳}\\
    \vspace{5mm}

    {\Large عنوان پروژه: سامانه آموزشی \lr{(LMS)}}\\

    \vspace{12mm}
    {\Large استاد: علی ابریشمی}\\
    \vspace{20mm}
    {\large طراحان پروژه: }\\
    \vspace{4mm}
    {\large مانی ابراهیمی}\\
    {\large نیما پشابادی}\\
    {\large محمدامین حیدری}\\
    {\large سید امیرمحمد جزایری}\\

    \vspace{20mm}
    {\Large تاریخ تحویل: ۱۱ بهمن ۱۴۰۳}\\
\end{center}

\newpage
\tableofcontents

\newpage
\chapter{معرفی پروژه}
\section{شرح پروژه}

در این پروژه قرار است شما یک \lr{LMS} یا سامانه مدیریت یادگیری بسازید، یعنی سامانه‌ای مشابه با \lr{CW} شریف یا \lr{Courses} امیرکبیر که خوب است بدانید برپایه‌ی \lr{Moodle}\footnote{\href{https://moodle.org}{\url{https://moodle.org}}} ساخته شده‌اند. 
در این راستا، شما مدل‌های سامانه را تعریف خواهید کرد و با ایجاد ارتباط منطقی میان آن‌ها یک سامانه‌ی قابل اجرا خواهید داشت. همچنین در بخش‌هایی از این پروژه از شما خواسته خواهد شد تا با استفاده از \lr{Pandas} و \lr{Numpy} داده‌های کلاسی را پردازش کرده و یک خروجی خوانا برای کاربران سامانه ایجاد کنید.

\section{انتظارات و اهداف}

انتظار می‌رود در انتهای این پروژه:

\begin{itemize}
    \item {
        بتوانید با استفاده از \lr{Pandas} و \lr{Numpy} داده‌ها را پردازش کرده و یک خروجی خوانا برای کاربران سامانه ایجاد کنید.
    }
    \item {
        تا حد بسیار مبتدی با مفاهیم ذخیره‌سازی داده‌ها و پردازش آن‌ها آشنا شوید.
    }
    \item {
        یک سیستم را با استفاده از دانسته‌های فعلی خود تولید کرده و توسعه دهید.
    }
\end{itemize}
\noindent
و همچنین در صورت انجام بخش اختیاری هم از شما انتظار می‌رود:

\begin{itemize}
    \item {
        رویکرد شی‌گرا را به خوبی آموخته و بتوانید با آن یک پروژه‌ی قابل اجرا بسازید.
    }
    \item {
        بتوانید با \lr{Git} تغییرات یک پروژه را مدیریت کنید. همچنین با گیت‌هاب آشنا شوید و بتوانید پروژه خود را به اشتراک بگذارید.
    }
    \item {
        با مفهوم \lr{GUI} آشنا شوید و بتوانید با استفاده از \lr{Tkinter} یا \lr{PyQt} یک پنجره‌ی \lr{GUI} بسازید.
    }
    \item {
        با اهمیت آزمایش آشنا شده و بتوانید برای یک پروژه \lr{Unit Test} بنویسید.
    }
\end{itemize}

\section{قوانین مربوط به پروژه}

\begin{itemize}
    \item {
        در این پروژه شما مجاز به استفاده از مدل‌های بزرگ زبانی\footnote{\lr{LLM}} مانند \lr{ChatGPT} یا \lr{Claude} هستید؛ به شرط اینکه هر دو عضو گروه تسلط کامل بر کد پروژه داشته باشند و بتوانند حین تحویل آن را به خوبی توضیح دهند.
    }
    \item {
        مستند پروژه را یک بار تا انتها بخوانید، تعدادی از موارد امتیازی را (در صورتی که تمایل به انـجام آن‌ها دارید) می‌بایست از همان ابتدای شروع پروژه انـجام دهید و در نظر داشته باشید.
    }
    \item {
        تمام فایل‌های پروژه‌تان را در یک فایل فشرده با \underline{فرمت \,\texttt{.zip}}  روی کوئرا بارگذاری کنید.
    }
    \item {
        نام فایلی که بارگذاری می‌شود باید به فرمت \texttt{FOP\_PROJ\_\{STDID1\}\_\{STDID2\}} باشد که \texttt{STDID1} شماره دانشجویی عضو اول گروه و \texttt{STDID2} شماره دانشجویی عضو دوم گروه است.
        برای مثال اگر یک گروه داشته باشیم که شماره دانشجویی اعضایش \texttt{403108123} و \texttt{403108987} باشد،
        باید فایلی به نام \texttt{FOP\_PROJ\_403108123\_403108987.zip} آپلود کنند.\\
        {\large \underline{آپلود هرگونه فایل با نامگذاری خارج از این چارچوب موجب کسر نمره خواهد شد}.}
    }
    \item {
        هر دو عضو گروه می‌بایست فایل مربوطه را بارگذاری نمایند. مسئولیت عدم بارگذاری یا بارگذاری فایل‌های متفاوت توسط اعضای یک گروه متوجه خود ایشان است.
    }
    \item {
        از آنجا که تحویل پروژه اجباری است، مطمئن شوید هر دو عضو گروه کاملاً بر پروژه مسلط باشند. همچنین هر دو عضو موظف به فعالیت هستند و در صورتی که یکی از اعضا فعالیت نکند، تمامی اعضا به عنوان یک گروه دچار کسر نمره خواهد شد.
    }
    \item {
        تبادل کد بین گروه‌ها اصلا مورد پذیرش نیست و در صورت تشخیص، نمره‌ی هر دو گروه دخیل کسر خواهد شد.
    }
    \item {
        اطمینان حاصل شده است که پروژه‌ی شما با تمامی مطالبی که در کلاس آموخته‌اید قابل انـجام باشد. پس پیش از انـجام پروژه، همه‌ی مطالبی که در کلاس آموخته‌اید را به خوبی مرور بفرمایید.
    }
\end{itemize}


\chapter{بخش اجباری (۲۰۰۰ نمره)}
\begin{center}
    \begin{warningbox}
        \Large
        جمع نمرات این بخش ۲۰۰۰ نمره است که معادل ۲ نمره از کل درس می‌باشد.
        انـجام موارد ذکر شده در این بخش اجباری است.
    \end{warningbox}
\end{center}

\section{تعریف کلاس برای مدل‌های مورد نیاز (۲۵۰ نمره)}

\section{ایجاد فرآیند ثبت‌نام و ورود (۱۰۰ نمره)}

\section{ذخیره‌ی اطلاعات کاربران (دانشجویان و اساتید) روی فایل (۱۰۰ نمره)}

\section{ساخت شیت لیست کلاسی با \lr{Pandas} (۱۵۰ نمره)}

\section{استخراج شیت کلاسی با فرمت \texttt{CSV} و \texttt{XLSX} (۵۰ نمره)}

\section{محاسبه نمره و اعمال نمودار روی نمره (۵۰ نمره)}

\section{پویا کردن داده‌ها با استفاده از فرمول‌های اکسل (۵۰ نمره)}

\chapter{بخش اختیاری (۵۰۰ نمره)}
\begin{center}
    \begin{warningbox}
        \Large
        جمع نمرات این بخش ۵۰۰ نمره است که معادل ۰٫۵ نمره از کل درس می‌باشد.
        انـجام موارد ذکر شده در این بخش اجباری نیست اما نمره‌ی امتیازی بر روی کل درس دارد.
    \end{warningbox}
\end{center}

\section{استفاده از \lr{Git} (۱۲۰ نمره)}

\section{ایجاد رابط کاربری گرافیکی (۱۳۰ نمره)}

\section{انتشار روی \lr{GitHub} (۵۰ نمره)}


\chapter{ضمیمه ۱: ماژول‌های قابل توجه}
\end{document}